% Options for packages loaded elsewhere
\PassOptionsToPackage{unicode}{hyperref}
\PassOptionsToPackage{hyphens}{url}
\documentclass[
  11pt,
]{article}
\usepackage{xcolor}
\usepackage[margin=1in]{geometry}
\usepackage{amsmath,amssymb}
\setcounter{secnumdepth}{5}
\usepackage{iftex}
\ifPDFTeX
  \usepackage[T1]{fontenc}
  \usepackage[utf8]{inputenc}
  \usepackage{textcomp} % provide euro and other symbols
\else % if luatex or xetex
  \usepackage{unicode-math} % this also loads fontspec
  \defaultfontfeatures{Scale=MatchLowercase}
  \defaultfontfeatures[\rmfamily]{Ligatures=TeX,Scale=1}
\fi
\usepackage{lmodern}
\ifPDFTeX\else
  % xetex/luatex font selection
\fi
% Use upquote if available, for straight quotes in verbatim environments
\IfFileExists{upquote.sty}{\usepackage{upquote}}{}
\IfFileExists{microtype.sty}{% use microtype if available
  \usepackage[]{microtype}
  \UseMicrotypeSet[protrusion]{basicmath} % disable protrusion for tt fonts
}{}
\makeatletter
\@ifundefined{KOMAClassName}{% if non-KOMA class
  \IfFileExists{parskip.sty}{%
    \usepackage{parskip}
  }{% else
    \setlength{\parindent}{0pt}
    \setlength{\parskip}{6pt plus 2pt minus 1pt}}
}{% if KOMA class
  \KOMAoptions{parskip=half}}
\makeatother
\usepackage{graphicx}
\makeatletter
\newsavebox\pandoc@box
\newcommand*\pandocbounded[1]{% scales image to fit in text height/width
  \sbox\pandoc@box{#1}%
  \Gscale@div\@tempa{\textheight}{\dimexpr\ht\pandoc@box+\dp\pandoc@box\relax}%
  \Gscale@div\@tempb{\linewidth}{\wd\pandoc@box}%
  \ifdim\@tempb\p@<\@tempa\p@\let\@tempa\@tempb\fi% select the smaller of both
  \ifdim\@tempa\p@<\p@\scalebox{\@tempa}{\usebox\pandoc@box}%
  \else\usebox{\pandoc@box}%
  \fi%
}
% Set default figure placement to htbp
\def\fps@figure{htbp}
\makeatother
\setlength{\emergencystretch}{3em} % prevent overfull lines
\providecommand{\tightlist}{%
  \setlength{\itemsep}{0pt}\setlength{\parskip}{0pt}}
\usepackage{booktabs}
\usepackage{longtable}
\usepackage{array}
\usepackage{multirow}
\usepackage{wrapfig}
\usepackage{float}
\usepackage{colortbl}
\usepackage{pdflscape}
\usepackage{tabu}
\usepackage{threeparttable}
\usepackage{threeparttablex}
\usepackage[normalem]{ulem}
\usepackage{makecell}
\usepackage{xcolor}
\usepackage{bookmark}
\IfFileExists{xurl.sty}{\usepackage{xurl}}{} % add URL line breaks if available
\urlstyle{same}
\hypersetup{
  pdftitle={Lista 03 -- Métodos de Preprocesamiento de Datos},
  pdfauthor={Víctor Mauricio Ochoa García, Salvador Enrique Rodríguez Hernández},
  hidelinks,
  pdfcreator={LaTeX via pandoc}}

\title{Lista 03 -- Métodos de Preprocesamiento de Datos}
\usepackage{etoolbox}
\makeatletter
\providecommand{\subtitle}[1]{% add subtitle to \maketitle
  \apptocmd{\@title}{\par {\large #1 \par}}{}{}
}
\makeatother
\subtitle{Análisis de datos y estrategias de balanceo}
\author{Víctor Mauricio Ochoa García, Salvador Enrique Rodríguez
Hernández}
\date{10 de noviembre de 2025}

\begin{document}
\maketitle

\newpage

\tableofcontents

\newpage

\section{\texorpdfstring{(a) Análisis descriptivo del conjunto
\emph{mtcars}}{(a) Análisis descriptivo del conjunto mtcars}}\label{a-anuxe1lisis-descriptivo-del-conjunto-mtcars}

\pandocbounded{\includegraphics[keepaspectratio]{Lista-03-de-ejercicios_files/figure-latex/unnamed-chunk-1-1.pdf}}
\pandocbounded{\includegraphics[keepaspectratio]{Lista-03-de-ejercicios_files/figure-latex/unnamed-chunk-1-2.pdf}}
\pandocbounded{\includegraphics[keepaspectratio]{Lista-03-de-ejercicios_files/figure-latex/unnamed-chunk-1-3.pdf}}
\pandocbounded{\includegraphics[keepaspectratio]{Lista-03-de-ejercicios_files/figure-latex/unnamed-chunk-1-4.pdf}}
\pandocbounded{\includegraphics[keepaspectratio]{Lista-03-de-ejercicios_files/figure-latex/unnamed-chunk-1-5.pdf}}

\section{(b) Selección de variables con el método
mRMR}\label{b-selecciuxf3n-de-variables-con-el-muxe9todo-mrmr}

\begin{table}
\centering
\caption{\label{tab:unnamed-chunk-2}Tabla 1. Relevancia de las variables predictoras respecto a mpg}
\centering
\begin{tabular}[t]{llr}
\toprule
Variable & Tipo & Medida de relevancia\\
\midrule
wt & Numérica & 0.8677\\
cyl & Numérica & 0.8522\\
disp & Numérica & 0.8476\\
hp & Numérica & 0.7762\\
drat & Numérica & 0.6812\\
\addlinespace
vs & Numérica & 0.6640\\
am & Numérica & 0.5998\\
carb & Numérica & 0.5509\\
gear & Numérica & 0.4803\\
qsec & Numérica & 0.4187\\
\bottomrule
\end{tabular}
\end{table}

\section{(c) Modelo de regresión lineal múltiple (todas las
variables)}\label{c-modelo-de-regresiuxf3n-lineal-muxfaltiple-todas-las-variables}

\begin{table}
\centering
\caption{\label{tab:unnamed-chunk-3}Coeficientes – Modelo completo}
\centering
\begin{tabular}[t]{lrrrr}
\toprule
Termino & Estimacion & Error.Estd & t & p.value\\
\midrule
(Intercept) & 35.9124 & 17.9703 & 1.9984 & 0.0710\\
cyl & -0.8185 & 1.2973 & -0.6309 & 0.5410\\
disp & -0.0007 & 0.0191 & -0.0389 & 0.9697\\
hp & -0.0214 & 0.0309 & -0.6907 & 0.5040\\
drat & 0.0167 & 2.7544 & 0.0061 & 0.9953\\
\addlinespace
wt & -1.5688 & 1.9870 & -0.7896 & 0.4465\\
qsec & -0.5079 & 0.8443 & -0.6016 & 0.5597\\
vs & 1.5084 & 3.1877 & 0.4732 & 0.6453\\
am & -1.4891 & 2.9109 & -0.5115 & 0.6191\\
gear & 1.8601 & 1.5868 & 1.1722 & 0.2659\\
\addlinespace
carb & -0.2998 & 1.0255 & -0.2923 & 0.7755\\
\bottomrule
\end{tabular}
\end{table}

\begin{table}
\centering
\caption{\label{tab:unnamed-chunk-3}Medidas de ajuste – Modelo completo}
\centering
\begin{tabular}[t]{rrrrr}
\toprule
R\textasciicircum{}2 (train) & R\textasciicircum{}2 ajustado (train) & ECM (test) & R\textasciicircum{}2 (test) & Sigma resid.\\
\midrule
0.9094 & 0.8271 & 17.788 & 0.6027 & 2.1797\\
\bottomrule
\end{tabular}
\end{table}

\section{(d) Modelo reducido con variables seleccionadas por
mRMR}\label{d-modelo-reducido-con-variables-seleccionadas-por-mrmr}

\begin{table}
\centering
\caption{\label{tab:unnamed-chunk-4}Coeficientes – Modelo reducido (mRMR)}
\centering
\begin{tabular}[t]{lrrrr}
\toprule
Termino & Estimacion & Error.Estd & t & p.value\\
\midrule
(Intercept) & 37.7718 & 2.5858 & 14.6073 & 0.0000\\
wt & -2.7809 & 0.9170 & -3.0324 & 0.0072\\
cyl & -1.6405 & 0.5140 & -3.1916 & 0.0051\\
disp & 0.0037 & 0.0111 & 0.3288 & 0.7461\\
\bottomrule
\end{tabular}
\end{table}

\begin{table}
\centering
\caption{\label{tab:unnamed-chunk-4}Medidas de ajuste – Modelo reducido (mRMR)}
\centering
\begin{tabular}[t]{rrrrr}
\toprule
R\textasciicircum{}2 (train) & R\textasciicircum{}2 ajustado (train) & ECM (test) & R\textasciicircum{}2 (test) & Sigma resid.\\
\midrule
0.8878 & 0.8691 & 14.6777 & 0.6722 & 1.8961\\
\bottomrule
\end{tabular}
\end{table}

\section{(e) Comparación entre
modelos}\label{e-comparaciuxf3n-entre-modelos}

\begin{table}
\centering
\caption{\label{tab:unnamed-chunk-5}Comparación de desempeño entre modelos}
\centering
\begin{tabular}[t]{lrr}
\toprule
Modelo & ECM & R\textasciicircum{}2\\
\midrule
Completo (todas las variables) & 17.7880 & 0.6027\\
Reducido (3 variables mRMR) & 14.6777 & 0.6722\\
\bottomrule
\end{tabular}
\end{table}

\end{document}
