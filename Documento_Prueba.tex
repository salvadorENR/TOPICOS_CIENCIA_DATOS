% Options for packages loaded elsewhere
\PassOptionsToPackage{unicode}{hyperref}
\PassOptionsToPackage{hyphens}{url}
\documentclass[
  11pt,
]{article}
\usepackage{xcolor}
\usepackage[margin=1in]{geometry}
\usepackage{amsmath,amssymb}
\setcounter{secnumdepth}{5}
\usepackage{iftex}
\ifPDFTeX
  \usepackage[T1]{fontenc}
  \usepackage[utf8]{inputenc}
  \usepackage{textcomp} % provide euro and other symbols
\else % if luatex or xetex
  \usepackage{unicode-math} % this also loads fontspec
  \defaultfontfeatures{Scale=MatchLowercase}
  \defaultfontfeatures[\rmfamily]{Ligatures=TeX,Scale=1}
\fi
\usepackage{lmodern}
\ifPDFTeX\else
  % xetex/luatex font selection
\fi
% Use upquote if available, for straight quotes in verbatim environments
\IfFileExists{upquote.sty}{\usepackage{upquote}}{}
\IfFileExists{microtype.sty}{% use microtype if available
  \usepackage[]{microtype}
  \UseMicrotypeSet[protrusion]{basicmath} % disable protrusion for tt fonts
}{}
\makeatletter
\@ifundefined{KOMAClassName}{% if non-KOMA class
  \IfFileExists{parskip.sty}{%
    \usepackage{parskip}
  }{% else
    \setlength{\parindent}{0pt}
    \setlength{\parskip}{6pt plus 2pt minus 1pt}}
}{% if KOMA class
  \KOMAoptions{parskip=half}}
\makeatother
\usepackage{color}
\usepackage{fancyvrb}
\newcommand{\VerbBar}{|}
\newcommand{\VERB}{\Verb[commandchars=\\\{\}]}
\DefineVerbatimEnvironment{Highlighting}{Verbatim}{commandchars=\\\{\}}
% Add ',fontsize=\small' for more characters per line
\usepackage{framed}
\definecolor{shadecolor}{RGB}{248,248,248}
\newenvironment{Shaded}{\begin{snugshade}}{\end{snugshade}}
\newcommand{\AlertTok}[1]{\textcolor[rgb]{0.94,0.16,0.16}{#1}}
\newcommand{\AnnotationTok}[1]{\textcolor[rgb]{0.56,0.35,0.01}{\textbf{\textit{#1}}}}
\newcommand{\AttributeTok}[1]{\textcolor[rgb]{0.13,0.29,0.53}{#1}}
\newcommand{\BaseNTok}[1]{\textcolor[rgb]{0.00,0.00,0.81}{#1}}
\newcommand{\BuiltInTok}[1]{#1}
\newcommand{\CharTok}[1]{\textcolor[rgb]{0.31,0.60,0.02}{#1}}
\newcommand{\CommentTok}[1]{\textcolor[rgb]{0.56,0.35,0.01}{\textit{#1}}}
\newcommand{\CommentVarTok}[1]{\textcolor[rgb]{0.56,0.35,0.01}{\textbf{\textit{#1}}}}
\newcommand{\ConstantTok}[1]{\textcolor[rgb]{0.56,0.35,0.01}{#1}}
\newcommand{\ControlFlowTok}[1]{\textcolor[rgb]{0.13,0.29,0.53}{\textbf{#1}}}
\newcommand{\DataTypeTok}[1]{\textcolor[rgb]{0.13,0.29,0.53}{#1}}
\newcommand{\DecValTok}[1]{\textcolor[rgb]{0.00,0.00,0.81}{#1}}
\newcommand{\DocumentationTok}[1]{\textcolor[rgb]{0.56,0.35,0.01}{\textbf{\textit{#1}}}}
\newcommand{\ErrorTok}[1]{\textcolor[rgb]{0.64,0.00,0.00}{\textbf{#1}}}
\newcommand{\ExtensionTok}[1]{#1}
\newcommand{\FloatTok}[1]{\textcolor[rgb]{0.00,0.00,0.81}{#1}}
\newcommand{\FunctionTok}[1]{\textcolor[rgb]{0.13,0.29,0.53}{\textbf{#1}}}
\newcommand{\ImportTok}[1]{#1}
\newcommand{\InformationTok}[1]{\textcolor[rgb]{0.56,0.35,0.01}{\textbf{\textit{#1}}}}
\newcommand{\KeywordTok}[1]{\textcolor[rgb]{0.13,0.29,0.53}{\textbf{#1}}}
\newcommand{\NormalTok}[1]{#1}
\newcommand{\OperatorTok}[1]{\textcolor[rgb]{0.81,0.36,0.00}{\textbf{#1}}}
\newcommand{\OtherTok}[1]{\textcolor[rgb]{0.56,0.35,0.01}{#1}}
\newcommand{\PreprocessorTok}[1]{\textcolor[rgb]{0.56,0.35,0.01}{\textit{#1}}}
\newcommand{\RegionMarkerTok}[1]{#1}
\newcommand{\SpecialCharTok}[1]{\textcolor[rgb]{0.81,0.36,0.00}{\textbf{#1}}}
\newcommand{\SpecialStringTok}[1]{\textcolor[rgb]{0.31,0.60,0.02}{#1}}
\newcommand{\StringTok}[1]{\textcolor[rgb]{0.31,0.60,0.02}{#1}}
\newcommand{\VariableTok}[1]{\textcolor[rgb]{0.00,0.00,0.00}{#1}}
\newcommand{\VerbatimStringTok}[1]{\textcolor[rgb]{0.31,0.60,0.02}{#1}}
\newcommand{\WarningTok}[1]{\textcolor[rgb]{0.56,0.35,0.01}{\textbf{\textit{#1}}}}
\usepackage{longtable,booktabs,array}
\usepackage{calc} % for calculating minipage widths
% Correct order of tables after \paragraph or \subparagraph
\usepackage{etoolbox}
\makeatletter
\patchcmd\longtable{\par}{\if@noskipsec\mbox{}\fi\par}{}{}
\makeatother
% Allow footnotes in longtable head/foot
\IfFileExists{footnotehyper.sty}{\usepackage{footnotehyper}}{\usepackage{footnote}}
\makesavenoteenv{longtable}
\usepackage{graphicx}
\makeatletter
\newsavebox\pandoc@box
\newcommand*\pandocbounded[1]{% scales image to fit in text height/width
  \sbox\pandoc@box{#1}%
  \Gscale@div\@tempa{\textheight}{\dimexpr\ht\pandoc@box+\dp\pandoc@box\relax}%
  \Gscale@div\@tempb{\linewidth}{\wd\pandoc@box}%
  \ifdim\@tempb\p@<\@tempa\p@\let\@tempa\@tempb\fi% select the smaller of both
  \ifdim\@tempa\p@<\p@\scalebox{\@tempa}{\usebox\pandoc@box}%
  \else\usebox{\pandoc@box}%
  \fi%
}
% Set default figure placement to htbp
\def\fps@figure{htbp}
\makeatother
\setlength{\emergencystretch}{3em} % prevent overfull lines
\providecommand{\tightlist}{%
  \setlength{\itemsep}{0pt}\setlength{\parskip}{0pt}}
\usepackage{bookmark}
\IfFileExists{xurl.sty}{\usepackage{xurl}}{} % add URL line breaks if available
\urlstyle{same}
\hypersetup{
  pdftitle={Lista 02 -- Métodos de Preprocesamiento de Datos},
  pdfauthor={Víctor Mauricio Ochoa García, Salvador Enrique Rodríguez Hernández},
  hidelinks,
  pdfcreator={LaTeX via pandoc}}

\title{Lista 02 -- Métodos de Preprocesamiento de Datos}
\usepackage{etoolbox}
\makeatletter
\providecommand{\subtitle}[1]{% add subtitle to \maketitle
  \apptocmd{\@title}{\par {\large #1 \par}}{}{}
}
\makeatother
\subtitle{Análisis de datos y estrategias de balanceo}
\author{Víctor Mauricio Ochoa García, Salvador Enrique Rodríguez
Hernández}
\date{06 de noviembre de 2025}

\begin{document}
\maketitle

\newpage

\tableofcontents

\newpage

\begin{Shaded}
\begin{Highlighting}[]
\NormalTok{knitr}\SpecialCharTok{::}\NormalTok{opts\_chunk}\SpecialCharTok{$}\FunctionTok{set}\NormalTok{(}
\AttributeTok{echo =} \ConstantTok{TRUE}\NormalTok{, }\AttributeTok{message =} \ConstantTok{FALSE}\NormalTok{, }\AttributeTok{warning =} \ConstantTok{FALSE}\NormalTok{, }\AttributeTok{fig.width =} \DecValTok{7}\NormalTok{, }\AttributeTok{fig.height =} \DecValTok{5}
\NormalTok{)}

\FunctionTok{suppressPackageStartupMessages}\NormalTok{(\{}
\FunctionTok{library}\NormalTok{(readr)}
\FunctionTok{library}\NormalTok{(dplyr)}
\FunctionTok{library}\NormalTok{(ggplot2)}
\FunctionTok{library}\NormalTok{(rsample)}
\FunctionTok{library}\NormalTok{(recipes)}
\FunctionTok{library}\NormalTok{(themis)   }\CommentTok{\# SMOTE}
\FunctionTok{library}\NormalTok{(FNN)      }\CommentTok{\# knn/ENN}
\FunctionTok{library}\NormalTok{(knitr)}
\FunctionTok{library}\NormalTok{(tibble)}
\NormalTok{\})}
\FunctionTok{set.seed}\NormalTok{(}\DecValTok{2026}\NormalTok{)}
\end{Highlighting}
\end{Shaded}

\section{Datos y utilidades}\label{datos-y-utilidades}

\begin{Shaded}
\begin{Highlighting}[]
\CommentTok{\# Dataset UCI (trata "?" como NA)}

\NormalTok{url }\OtherTok{\textless{}{-}} \StringTok{"https://archive.ics.uci.edu/ml/machine{-}learning{-}databases/00383/risk\_factors\_cervical\_cancer.csv"}
\NormalTok{cc    }\OtherTok{\textless{}{-}} \FunctionTok{read\_csv}\NormalTok{(url, }\AttributeTok{na =} \StringTok{"?"}\NormalTok{)}
\NormalTok{datos }\OtherTok{\textless{}{-}} \FunctionTok{as.data.frame}\NormalTok{(cc)}

\NormalTok{target\_var }\OtherTok{\textless{}{-}} \StringTok{"Biopsy"}
\FunctionTok{stopifnot}\NormalTok{(target\_var }\SpecialCharTok{\%in\%} \FunctionTok{names}\NormalTok{(datos))}

\CommentTok{\# Asegurar factor binario "0"/"1"}

\NormalTok{y }\OtherTok{\textless{}{-}}\NormalTok{ datos[[target\_var]]}
\ControlFlowTok{if}\NormalTok{ (}\SpecialCharTok{!}\FunctionTok{is.factor}\NormalTok{(y)) y }\OtherTok{\textless{}{-}} \FunctionTok{factor}\NormalTok{(}\FunctionTok{as.character}\NormalTok{(y))}
\ControlFlowTok{if}\NormalTok{ (}\SpecialCharTok{!}\FunctionTok{all}\NormalTok{(}\FunctionTok{levels}\NormalTok{(y) }\SpecialCharTok{\%in\%} \FunctionTok{c}\NormalTok{(}\StringTok{"0"}\NormalTok{,}\StringTok{"1"}\NormalTok{))) \{}
\NormalTok{y }\OtherTok{\textless{}{-}} \FunctionTok{factor}\NormalTok{(}\FunctionTok{as.character}\NormalTok{(y), }\AttributeTok{levels =} \FunctionTok{c}\NormalTok{(}\StringTok{"0"}\NormalTok{,}\StringTok{"1"}\NormalTok{))}
\NormalTok{\}}
\NormalTok{datos[[target\_var]] }\OtherTok{\textless{}{-}}\NormalTok{ y}
\end{Highlighting}
\end{Shaded}

\begin{Shaded}
\begin{Highlighting}[]
\CommentTok{\# Métricas (idénticas al estilo de clase)}

\NormalTok{calc\_metrics }\OtherTok{\textless{}{-}} \ControlFlowTok{function}\NormalTok{(y\_true, y\_pred) \{}
\NormalTok{tab }\OtherTok{\textless{}{-}} \FunctionTok{table}\NormalTok{(}\AttributeTok{Pred =}\NormalTok{ y\_pred, }\AttributeTok{Real =}\NormalTok{ y\_true)}
\NormalTok{TP }\OtherTok{\textless{}{-}} \FunctionTok{ifelse}\NormalTok{(}\StringTok{"1"} \SpecialCharTok{\%in\%} \FunctionTok{rownames}\NormalTok{(tab) }\SpecialCharTok{\&} \StringTok{"1"} \SpecialCharTok{\%in\%} \FunctionTok{colnames}\NormalTok{(tab), tab[}\StringTok{"1"}\NormalTok{,}\StringTok{"1"}\NormalTok{], }\DecValTok{0}\NormalTok{)}
\NormalTok{TN }\OtherTok{\textless{}{-}} \FunctionTok{ifelse}\NormalTok{(}\StringTok{"0"} \SpecialCharTok{\%in\%} \FunctionTok{rownames}\NormalTok{(tab) }\SpecialCharTok{\&} \StringTok{"0"} \SpecialCharTok{\%in\%} \FunctionTok{colnames}\NormalTok{(tab), tab[}\StringTok{"0"}\NormalTok{,}\StringTok{"0"}\NormalTok{], }\DecValTok{0}\NormalTok{)}
\NormalTok{FP }\OtherTok{\textless{}{-}} \FunctionTok{ifelse}\NormalTok{(}\StringTok{"1"} \SpecialCharTok{\%in\%} \FunctionTok{rownames}\NormalTok{(tab) }\SpecialCharTok{\&} \StringTok{"0"} \SpecialCharTok{\%in\%} \FunctionTok{colnames}\NormalTok{(tab), tab[}\StringTok{"1"}\NormalTok{,}\StringTok{"0"}\NormalTok{], }\DecValTok{0}\NormalTok{)}
\NormalTok{FN }\OtherTok{\textless{}{-}} \FunctionTok{ifelse}\NormalTok{(}\StringTok{"0"} \SpecialCharTok{\%in\%} \FunctionTok{rownames}\NormalTok{(tab) }\SpecialCharTok{\&} \StringTok{"1"} \SpecialCharTok{\%in\%} \FunctionTok{colnames}\NormalTok{(tab), tab[}\StringTok{"0"}\NormalTok{,}\StringTok{"1"}\NormalTok{], }\DecValTok{0}\NormalTok{)}

\NormalTok{acc   }\OtherTok{\textless{}{-}} \FunctionTok{ifelse}\NormalTok{((TP }\SpecialCharTok{+}\NormalTok{ TN }\SpecialCharTok{+}\NormalTok{ FP }\SpecialCharTok{+}\NormalTok{ FN) }\SpecialCharTok{\textgreater{}} \DecValTok{0}\NormalTok{, (TP }\SpecialCharTok{+}\NormalTok{ TN)}\SpecialCharTok{/}\NormalTok{(TP }\SpecialCharTok{+}\NormalTok{ TN }\SpecialCharTok{+}\NormalTok{ FP }\SpecialCharTok{+}\NormalTok{ FN), }\ConstantTok{NA}\NormalTok{)}
\NormalTok{sens  }\OtherTok{\textless{}{-}} \FunctionTok{ifelse}\NormalTok{((TP }\SpecialCharTok{+}\NormalTok{ FN) }\SpecialCharTok{\textgreater{}} \DecValTok{0}\NormalTok{, TP }\SpecialCharTok{/}\NormalTok{ (TP }\SpecialCharTok{+}\NormalTok{ FN), }\ConstantTok{NA}\NormalTok{)}
\NormalTok{esp   }\OtherTok{\textless{}{-}} \FunctionTok{ifelse}\NormalTok{((TN }\SpecialCharTok{+}\NormalTok{ FP) }\SpecialCharTok{\textgreater{}} \DecValTok{0}\NormalTok{, TN }\SpecialCharTok{/}\NormalTok{ (TN }\SpecialCharTok{+}\NormalTok{ FP), }\ConstantTok{NA}\NormalTok{)}
\NormalTok{ppv   }\OtherTok{\textless{}{-}} \FunctionTok{ifelse}\NormalTok{((TP }\SpecialCharTok{+}\NormalTok{ FP) }\SpecialCharTok{\textgreater{}} \DecValTok{0}\NormalTok{, TP }\SpecialCharTok{/}\NormalTok{ (TP }\SpecialCharTok{+}\NormalTok{ FP), }\ConstantTok{NA}\NormalTok{)}
\NormalTok{npv   }\OtherTok{\textless{}{-}} \FunctionTok{ifelse}\NormalTok{((TN }\SpecialCharTok{+}\NormalTok{ FN) }\SpecialCharTok{\textgreater{}} \DecValTok{0}\NormalTok{, TN }\SpecialCharTok{/}\NormalTok{ (TN }\SpecialCharTok{+}\NormalTok{ FN), }\ConstantTok{NA}\NormalTok{)}
\NormalTok{gmean }\OtherTok{\textless{}{-}} \FunctionTok{ifelse}\NormalTok{(}\SpecialCharTok{!}\FunctionTok{is.na}\NormalTok{(sens) }\SpecialCharTok{\&} \SpecialCharTok{!}\FunctionTok{is.na}\NormalTok{(esp), }\FunctionTok{sqrt}\NormalTok{(sens }\SpecialCharTok{*}\NormalTok{ esp), }\ConstantTok{NA}\NormalTok{)}
\NormalTok{f1    }\OtherTok{\textless{}{-}} \FunctionTok{ifelse}\NormalTok{((ppv }\SpecialCharTok{+}\NormalTok{ sens) }\SpecialCharTok{\textgreater{}} \DecValTok{0}\NormalTok{, }\DecValTok{2} \SpecialCharTok{*}\NormalTok{ ppv }\SpecialCharTok{*}\NormalTok{ sens }\SpecialCharTok{/}\NormalTok{ (ppv }\SpecialCharTok{+}\NormalTok{ sens), }\ConstantTok{NA}\NormalTok{)}

\NormalTok{denom }\OtherTok{\textless{}{-}} \FunctionTok{sqrt}\NormalTok{( (TP }\SpecialCharTok{+}\NormalTok{ FP) }\SpecialCharTok{*}\NormalTok{ (TP }\SpecialCharTok{+}\NormalTok{ FN) }\SpecialCharTok{*}\NormalTok{ (TN }\SpecialCharTok{+}\NormalTok{ FP) }\SpecialCharTok{*}\NormalTok{ (TN }\SpecialCharTok{+}\NormalTok{ FN) )}
\NormalTok{mcc   }\OtherTok{\textless{}{-}} \FunctionTok{ifelse}\NormalTok{(denom }\SpecialCharTok{\textgreater{}} \DecValTok{0}\NormalTok{, (TP}\SpecialCharTok{*}\NormalTok{TN }\SpecialCharTok{{-}}\NormalTok{ FP}\SpecialCharTok{*}\NormalTok{FN) }\SpecialCharTok{/}\NormalTok{ denom, }\ConstantTok{NA}\NormalTok{)}

\FunctionTok{data.frame}\NormalTok{(}\AttributeTok{Accuracy =}\NormalTok{ acc, }\AttributeTok{Sensitivity =}\NormalTok{ sens, }\AttributeTok{Specificity =}\NormalTok{ esp,}
\AttributeTok{PPV =}\NormalTok{ ppv, }\AttributeTok{NPV =}\NormalTok{ npv, }\AttributeTok{Gmean =}\NormalTok{ gmean, }\AttributeTok{F1 =}\NormalTok{ f1, }\AttributeTok{MCC =}\NormalTok{ mcc)}
\NormalTok{\}}

\CommentTok{\# ENN manual (edita ejemplos de la clase mayoritaria que discrepan con vecinos)}

\NormalTok{ENN\_manual }\OtherTok{\textless{}{-}} \ControlFlowTok{function}\NormalTok{(data, target, }\AttributeTok{k =} \DecValTok{3}\NormalTok{, majority\_class) \{}
\NormalTok{X }\OtherTok{\textless{}{-}}\NormalTok{ data[, }\FunctionTok{setdiff}\NormalTok{(}\FunctionTok{names}\NormalTok{(data), target), drop }\OtherTok{=} \ConstantTok{FALSE}\NormalTok{]}
\NormalTok{y }\OtherTok{\textless{}{-}}\NormalTok{ data[[target]]}
\NormalTok{X\_num }\OtherTok{\textless{}{-}} \FunctionTok{as.data.frame}\NormalTok{(}\FunctionTok{lapply}\NormalTok{(X, }\ControlFlowTok{function}\NormalTok{(z) }\ControlFlowTok{if}\NormalTok{(}\FunctionTok{is.numeric}\NormalTok{(z)) z }\ControlFlowTok{else} \FunctionTok{as.numeric}\NormalTok{(}\FunctionTok{as.factor}\NormalTok{(z))))}
\NormalTok{nn }\OtherTok{\textless{}{-}} \FunctionTok{knnx.index}\NormalTok{(}\FunctionTok{as.matrix}\NormalTok{(X\_num), }\FunctionTok{as.matrix}\NormalTok{(X\_num), }\AttributeTok{k =}\NormalTok{ k }\SpecialCharTok{+} \DecValTok{1}\NormalTok{)[, }\SpecialCharTok{{-}}\DecValTok{1}\NormalTok{]}
\NormalTok{remove\_idx }\OtherTok{\textless{}{-}} \FunctionTok{sapply}\NormalTok{(}\DecValTok{1}\SpecialCharTok{:}\FunctionTok{nrow}\NormalTok{(X\_num), }\ControlFlowTok{function}\NormalTok{(i) \{}
\ControlFlowTok{if}\NormalTok{ (y[i] }\SpecialCharTok{==}\NormalTok{ majority\_class) \{}
\NormalTok{neigh\_classes }\OtherTok{\textless{}{-}}\NormalTok{ y[nn[i, ]]}
\NormalTok{maj\_class }\OtherTok{\textless{}{-}} \FunctionTok{names}\NormalTok{(}\FunctionTok{sort}\NormalTok{(}\FunctionTok{table}\NormalTok{(neigh\_classes), }\AttributeTok{decreasing =} \ConstantTok{TRUE}\NormalTok{))[}\DecValTok{1}\NormalTok{]}
\FunctionTok{return}\NormalTok{(maj\_class }\SpecialCharTok{!=}\NormalTok{ y[i])}
\NormalTok{\} }\ControlFlowTok{else} \ConstantTok{FALSE}
\NormalTok{\})}
\NormalTok{data[}\SpecialCharTok{!}\NormalTok{remove\_idx, , drop }\OtherTok{=} \ConstantTok{FALSE}\NormalTok{]}
\NormalTok{\}}

\CommentTok{\# Helper: ajustar y evaluar regresión logística}

\NormalTok{eval\_logit }\OtherTok{\textless{}{-}} \ControlFlowTok{function}\NormalTok{(train\_df, test\_df, }\AttributeTok{cutoff =} \FloatTok{0.5}\NormalTok{, }\AttributeTok{target =}\NormalTok{ target\_var) \{}
\NormalTok{form }\OtherTok{\textless{}{-}} \FunctionTok{as.formula}\NormalTok{(}\FunctionTok{paste}\NormalTok{(target, }\StringTok{"\textasciitilde{} ."}\NormalTok{))}
\NormalTok{mod  }\OtherTok{\textless{}{-}} \FunctionTok{glm}\NormalTok{(form, }\AttributeTok{data =}\NormalTok{ train\_df, }\AttributeTok{family =}\NormalTok{ binomial)}
\NormalTok{prob }\OtherTok{\textless{}{-}} \FunctionTok{predict}\NormalTok{(mod, }\AttributeTok{newdata =}\NormalTok{ test\_df, }\AttributeTok{type =} \StringTok{"response"}\NormalTok{)}
\NormalTok{pred }\OtherTok{\textless{}{-}} \FunctionTok{ifelse}\NormalTok{(prob }\SpecialCharTok{\textgreater{}=}\NormalTok{ cutoff, }\StringTok{"1"}\NormalTok{, }\StringTok{"0"}\NormalTok{)}
\NormalTok{pred }\OtherTok{\textless{}{-}} \FunctionTok{factor}\NormalTok{(pred, }\AttributeTok{levels =} \FunctionTok{c}\NormalTok{(}\StringTok{"0"}\NormalTok{,}\StringTok{"1"}\NormalTok{))}
\FunctionTok{list}\NormalTok{(}\AttributeTok{metrics =} \FunctionTok{calc\_metrics}\NormalTok{(test\_df[[target]], pred),}
\AttributeTok{cutoff =}\NormalTok{ cutoff,}
\AttributeTok{n\_train =} \FunctionTok{nrow}\NormalTok{(train\_df))}
\NormalTok{\}}
\end{Highlighting}
\end{Shaded}

\section{(a) ¿Por qué imputación kNN en todas las predictoras con
NA?}\label{a-por-quuxe9-imputaciuxf3n-knn-en-todas-las-predictoras-con-na}

\begin{Shaded}
\begin{Highlighting}[]
\NormalTok{na\_counts }\OtherTok{\textless{}{-}} \FunctionTok{sapply}\NormalTok{(datos, }\ControlFlowTok{function}\NormalTok{(v) }\FunctionTok{sum}\NormalTok{(}\FunctionTok{is.na}\NormalTok{(v)))}
\NormalTok{na\_pct    }\OtherTok{\textless{}{-}} \FunctionTok{round}\NormalTok{(}\DecValTok{100} \SpecialCharTok{*}\NormalTok{ na\_counts }\SpecialCharTok{/} \FunctionTok{nrow}\NormalTok{(datos), }\DecValTok{2}\NormalTok{)}
\NormalTok{na\_tbl    }\OtherTok{\textless{}{-}} \FunctionTok{data.frame}\NormalTok{(}\AttributeTok{var =} \FunctionTok{names}\NormalTok{(na\_counts), }\AttributeTok{NA\_count =}\NormalTok{ na\_counts, }\AttributeTok{NA\_percent =}\NormalTok{ na\_pct) }\SpecialCharTok{|\textgreater{}}
\FunctionTok{arrange}\NormalTok{(}\FunctionTok{desc}\NormalTok{(NA\_percent))}
\FunctionTok{kable}\NormalTok{(}\FunctionTok{head}\NormalTok{(na\_tbl, }\DecValTok{12}\NormalTok{), }\AttributeTok{caption =} \StringTok{"Top 12 variables con mayor porcentaje de NA"}\NormalTok{)}
\end{Highlighting}
\end{Shaded}

\begin{longtable}[]{@{}
  >{\raggedright\arraybackslash}p{(\linewidth - 6\tabcolsep) * \real{0.3889}}
  >{\raggedright\arraybackslash}p{(\linewidth - 6\tabcolsep) * \real{0.3889}}
  >{\raggedleft\arraybackslash}p{(\linewidth - 6\tabcolsep) * \real{0.1000}}
  >{\raggedleft\arraybackslash}p{(\linewidth - 6\tabcolsep) * \real{0.1222}}@{}}
\caption{Top 12 variables con mayor porcentaje de NA}\tabularnewline
\toprule\noalign{}
\begin{minipage}[b]{\linewidth}\raggedright
\end{minipage} & \begin{minipage}[b]{\linewidth}\raggedright
var
\end{minipage} & \begin{minipage}[b]{\linewidth}\raggedleft
NA\_count
\end{minipage} & \begin{minipage}[b]{\linewidth}\raggedleft
NA\_percent
\end{minipage} \\
\midrule\noalign{}
\endfirsthead
\toprule\noalign{}
\begin{minipage}[b]{\linewidth}\raggedright
\end{minipage} & \begin{minipage}[b]{\linewidth}\raggedright
var
\end{minipage} & \begin{minipage}[b]{\linewidth}\raggedleft
NA\_count
\end{minipage} & \begin{minipage}[b]{\linewidth}\raggedleft
NA\_percent
\end{minipage} \\
\midrule\noalign{}
\endhead
\bottomrule\noalign{}
\endlastfoot
STDs: Time since first diagnosis & STDs: Time since first diagnosis &
787 & 91.72 \\
STDs: Time since last diagnosis & STDs: Time since last diagnosis & 787
& 91.72 \\
IUD & IUD & 117 & 13.64 \\
IUD (years) & IUD (years) & 117 & 13.64 \\
Hormonal Contraceptives & Hormonal Contraceptives & 108 & 12.59 \\
Hormonal Contraceptives (years) & Hormonal Contraceptives (years) & 108
& 12.59 \\
STDs & STDs & 105 & 12.24 \\
STDs (number) & STDs (number) & 105 & 12.24 \\
STDs:condylomatosis & STDs:condylomatosis & 105 & 12.24 \\
STDs:cervical condylomatosis & STDs:cervical condylomatosis & 105 &
12.24 \\
STDs:vaginal condylomatosis & STDs:vaginal condylomatosis & 105 &
12.24 \\
STDs:vulvo-perineal condylomatosis & STDs:vulvo-perineal condylomatosis
& 105 & 12.24 \\
\end{longtable}

Respuesta (a). Varias variables presentan faltantes considerables (dos
por arriba de 90\% y varias en 10--15\%). Eliminar filas con NA
reduciría drásticamente el tamaño muestral y sesgaría el análisis. La
imputación por k-vecinos (kNN) aprovecha la estructura multivariada:
cada NA se reemplaza usando observaciones similares en el espacio de
variables, preservando coherencia entre predictores. Por ello se decidió
imputar con kNN en todas las predictoras con NA (y luego estandarizar).

\section{(b) Proporción de clases de la respuesta (barras con
porcentajes)}\label{b-proporciuxf3n-de-clases-de-la-respuesta-barras-con-porcentajes}

\begin{Shaded}
\begin{Highlighting}[]
\NormalTok{g\_base }\OtherTok{\textless{}{-}}\NormalTok{ datos }\SpecialCharTok{\%\textgreater{}\%}
\FunctionTok{count}\NormalTok{(}\SpecialCharTok{!!}\FunctionTok{sym}\NormalTok{(target\_var)) }\SpecialCharTok{\%\textgreater{}\%}
\FunctionTok{mutate}\NormalTok{(}\AttributeTok{prop =} \DecValTok{100} \SpecialCharTok{*}\NormalTok{ n }\SpecialCharTok{/} \FunctionTok{sum}\NormalTok{(n))}
\FunctionTok{kable}\NormalTok{(g\_base, }\AttributeTok{caption =} \StringTok{"Frecuencia y porcentaje por clase de Biopsy"}\NormalTok{)}
\end{Highlighting}
\end{Shaded}

\begin{longtable}[]{@{}lrr@{}}
\caption{Frecuencia y porcentaje por clase de Biopsy}\tabularnewline
\toprule\noalign{}
Biopsy & n & prop \\
\midrule\noalign{}
\endfirsthead
\toprule\noalign{}
Biopsy & n & prop \\
\midrule\noalign{}
\endhead
\bottomrule\noalign{}
\endlastfoot
0 & 803 & 93.589744 \\
1 & 55 & 6.410256 \\
\end{longtable}

\begin{Shaded}
\begin{Highlighting}[]
\FunctionTok{ggplot}\NormalTok{(g\_base, }\FunctionTok{aes}\NormalTok{(}\AttributeTok{x =} \SpecialCharTok{!!}\FunctionTok{sym}\NormalTok{(target\_var), }\AttributeTok{y =}\NormalTok{ n, }\AttributeTok{fill =} \SpecialCharTok{!!}\FunctionTok{sym}\NormalTok{(target\_var))) }\SpecialCharTok{+}
\FunctionTok{geom\_col}\NormalTok{(}\AttributeTok{width =} \FloatTok{0.6}\NormalTok{, }\AttributeTok{show.legend =} \ConstantTok{FALSE}\NormalTok{) }\SpecialCharTok{+}
\FunctionTok{geom\_text}\NormalTok{(}\FunctionTok{aes}\NormalTok{(}\AttributeTok{label =} \FunctionTok{paste0}\NormalTok{(}\FunctionTok{round}\NormalTok{(prop,}\DecValTok{1}\NormalTok{), }\StringTok{"\%"}\NormalTok{)), }\AttributeTok{vjust =} \SpecialCharTok{{-}}\FloatTok{0.3}\NormalTok{, }\AttributeTok{size =} \FloatTok{4.2}\NormalTok{) }\SpecialCharTok{+}
\FunctionTok{labs}\NormalTok{(}\AttributeTok{x =} \StringTok{"Clase (Biopsy)"}\NormalTok{, }\AttributeTok{y =} \StringTok{"Frecuencia absoluta"}\NormalTok{,}
\AttributeTok{title =} \StringTok{"Distribución de la respuesta (conteos y porcentajes)"}\NormalTok{) }\SpecialCharTok{+}
\FunctionTok{theme\_minimal}\NormalTok{(}\AttributeTok{base\_size =} \DecValTok{12}\NormalTok{)}
\end{Highlighting}
\end{Shaded}

\pandocbounded{\includegraphics[keepaspectratio]{Documento_Prueba_files/figure-latex/unnamed-chunk-5-1.pdf}}

La clase positiva (1) representa r sprintf(`\%.1f\%\%',
min(g\_base\$prop)) del total: el problema está fuertemente
desbalanceado.

Preparación común (imputar, normalizar, splits)

\begin{Shaded}
\begin{Highlighting}[]
\CommentTok{\# Split estratificado 70/30}

\NormalTok{split\_obj }\OtherTok{\textless{}{-}} \FunctionTok{initial\_split}\NormalTok{(datos, }\AttributeTok{prop =} \FloatTok{0.70}\NormalTok{, }\AttributeTok{strata =} \FunctionTok{all\_of}\NormalTok{(target\_var))}
\NormalTok{treino0 }\OtherTok{\textless{}{-}} \FunctionTok{training}\NormalTok{(split\_obj)}
\NormalTok{teste0  }\OtherTok{\textless{}{-}} \FunctionTok{testing}\NormalTok{(split\_obj)}

\CommentTok{\# Imputar primero, normalizar después (evita warnings y es el orden correcto)}

\NormalTok{rec\_imp }\OtherTok{\textless{}{-}} \FunctionTok{recipe}\NormalTok{(}\FunctionTok{as.formula}\NormalTok{(}\FunctionTok{paste}\NormalTok{(target\_var, }\StringTok{"\textasciitilde{} ."}\NormalTok{)), }\AttributeTok{data =}\NormalTok{ treino0) }\SpecialCharTok{\%\textgreater{}\%}
\FunctionTok{step\_impute\_knn}\NormalTok{(}\FunctionTok{all\_numeric\_predictors}\NormalTok{(), }\AttributeTok{neighbors =} \DecValTok{5}\NormalTok{) }\SpecialCharTok{\%\textgreater{}\%}
\FunctionTok{step\_zv}\NormalTok{(}\FunctionTok{all\_predictors}\NormalTok{()) }\SpecialCharTok{\%\textgreater{}\%}                  \CommentTok{\# quita predictores constantes}
\FunctionTok{step\_normalize}\NormalTok{(}\FunctionTok{all\_numeric\_predictors}\NormalTok{())}

\NormalTok{imp\_prep   }\OtherTok{\textless{}{-}} \FunctionTok{prep}\NormalTok{(rec\_imp)}
\NormalTok{treino\_imp }\OtherTok{\textless{}{-}} \FunctionTok{bake}\NormalTok{(imp\_prep, }\AttributeTok{new\_data =} \ConstantTok{NULL}\NormalTok{)}
\NormalTok{teste\_imp  }\OtherTok{\textless{}{-}} \FunctionTok{bake}\NormalTok{(imp\_prep, }\AttributeTok{new\_data =}\NormalTok{ teste0)}

\NormalTok{minority\_prop }\OtherTok{\textless{}{-}} \FunctionTok{min}\NormalTok{(g\_base}\SpecialCharTok{$}\NormalTok{prop }\SpecialCharTok{/} \DecValTok{100}\NormalTok{)}
\end{Highlighting}
\end{Shaded}

\section{(c) SMOTE + Regresión logística (corte
0.5)}\label{c-smote-regresiuxf3n-loguxedstica-corte-0.5}

\begin{Shaded}
\begin{Highlighting}[]
\NormalTok{rec\_smote }\OtherTok{\textless{}{-}} \FunctionTok{recipe}\NormalTok{(}\FunctionTok{as.formula}\NormalTok{(}\FunctionTok{paste}\NormalTok{(target\_var, }\StringTok{"\textasciitilde{} ."}\NormalTok{)), }\AttributeTok{data =}\NormalTok{ treino\_imp) }\SpecialCharTok{\%\textgreater{}\%}
\FunctionTok{step\_smote}\NormalTok{(}\FunctionTok{all\_outcomes}\NormalTok{()) }\SpecialCharTok{\%\textgreater{}\%}
\FunctionTok{step\_zv}\NormalTok{(}\FunctionTok{all\_predictors}\NormalTok{())}
\NormalTok{smote\_prep   }\OtherTok{\textless{}{-}} \FunctionTok{prep}\NormalTok{(rec\_smote)}
\NormalTok{treino\_smote }\OtherTok{\textless{}{-}} \FunctionTok{bake}\NormalTok{(smote\_prep, }\AttributeTok{new\_data =} \ConstantTok{NULL}\NormalTok{)}

\NormalTok{res\_c  }\OtherTok{\textless{}{-}} \FunctionTok{eval\_logit}\NormalTok{(treino\_smote, teste\_imp, }\AttributeTok{cutoff =} \FloatTok{0.5}\NormalTok{)}
\NormalTok{row\_c  }\OtherTok{\textless{}{-}} \FunctionTok{cbind}\NormalTok{(}\AttributeTok{Modelo =} \StringTok{"SMOTE (cut=0.5)"}\NormalTok{, }\AttributeTok{Ntrain =}\NormalTok{ res\_c}\SpecialCharTok{$}\NormalTok{n\_train, }\AttributeTok{Cutoff =}\NormalTok{ res\_c}\SpecialCharTok{$}\NormalTok{cutoff, res\_c}\SpecialCharTok{$}\NormalTok{metrics)}
\FunctionTok{kable}\NormalTok{(row\_c, }\AttributeTok{caption =} \StringTok{"Resultados (c): SMOTE + logística (cut=0.5)"}\NormalTok{)}
\end{Highlighting}
\end{Shaded}

\begin{longtable}[]{@{}
  >{\raggedright\arraybackslash}p{(\linewidth - 20\tabcolsep) * \real{0.1553}}
  >{\raggedleft\arraybackslash}p{(\linewidth - 20\tabcolsep) * \real{0.0680}}
  >{\raggedleft\arraybackslash}p{(\linewidth - 20\tabcolsep) * \real{0.0680}}
  >{\raggedleft\arraybackslash}p{(\linewidth - 20\tabcolsep) * \real{0.0971}}
  >{\raggedleft\arraybackslash}p{(\linewidth - 20\tabcolsep) * \real{0.1165}}
  >{\raggedleft\arraybackslash}p{(\linewidth - 20\tabcolsep) * \real{0.1165}}
  >{\raggedleft\arraybackslash}p{(\linewidth - 20\tabcolsep) * \real{0.0485}}
  >{\raggedleft\arraybackslash}p{(\linewidth - 20\tabcolsep) * \real{0.0971}}
  >{\raggedleft\arraybackslash}p{(\linewidth - 20\tabcolsep) * \real{0.0971}}
  >{\raggedleft\arraybackslash}p{(\linewidth - 20\tabcolsep) * \real{0.0388}}
  >{\raggedleft\arraybackslash}p{(\linewidth - 20\tabcolsep) * \real{0.0971}}@{}}
\caption{Resultados (c): SMOTE + logística (cut=0.5)}\tabularnewline
\toprule\noalign{}
\begin{minipage}[b]{\linewidth}\raggedright
Modelo
\end{minipage} & \begin{minipage}[b]{\linewidth}\raggedleft
Ntrain
\end{minipage} & \begin{minipage}[b]{\linewidth}\raggedleft
Cutoff
\end{minipage} & \begin{minipage}[b]{\linewidth}\raggedleft
Accuracy
\end{minipage} & \begin{minipage}[b]{\linewidth}\raggedleft
Sensitivity
\end{minipage} & \begin{minipage}[b]{\linewidth}\raggedleft
Specificity
\end{minipage} & \begin{minipage}[b]{\linewidth}\raggedleft
PPV
\end{minipage} & \begin{minipage}[b]{\linewidth}\raggedleft
NPV
\end{minipage} & \begin{minipage}[b]{\linewidth}\raggedleft
Gmean
\end{minipage} & \begin{minipage}[b]{\linewidth}\raggedleft
F1
\end{minipage} & \begin{minipage}[b]{\linewidth}\raggedleft
MCC
\end{minipage} \\
\midrule\noalign{}
\endfirsthead
\toprule\noalign{}
\begin{minipage}[b]{\linewidth}\raggedright
Modelo
\end{minipage} & \begin{minipage}[b]{\linewidth}\raggedleft
Ntrain
\end{minipage} & \begin{minipage}[b]{\linewidth}\raggedleft
Cutoff
\end{minipage} & \begin{minipage}[b]{\linewidth}\raggedleft
Accuracy
\end{minipage} & \begin{minipage}[b]{\linewidth}\raggedleft
Sensitivity
\end{minipage} & \begin{minipage}[b]{\linewidth}\raggedleft
Specificity
\end{minipage} & \begin{minipage}[b]{\linewidth}\raggedleft
PPV
\end{minipage} & \begin{minipage}[b]{\linewidth}\raggedleft
NPV
\end{minipage} & \begin{minipage}[b]{\linewidth}\raggedleft
Gmean
\end{minipage} & \begin{minipage}[b]{\linewidth}\raggedleft
F1
\end{minipage} & \begin{minipage}[b]{\linewidth}\raggedleft
MCC
\end{minipage} \\
\midrule\noalign{}
\endhead
\bottomrule\noalign{}
\endlastfoot
SMOTE (cut=0.5) & 1120 & 0.5 & 0.9534884 & 0.9333333 & 0.9547325 & 0.56
& 0.9957082 & 0.9439723 & 0.7 & 0.7024994 \\
\end{longtable}

SMOTE aumenta la presencia de la clase minoritaria en entrenamiento y la
logística mejora el equilibrio sensibilidad--especificidad.

\section{(d) ENN + Regresión logística (corte = prop.
minoritaria)}\label{d-enn-regresiuxf3n-loguxedstica-corte-prop.-minoritaria}

\begin{Shaded}
\begin{Highlighting}[]
\NormalTok{maj\_class  }\OtherTok{\textless{}{-}} \FunctionTok{names}\NormalTok{(}\FunctionTok{sort}\NormalTok{(}\FunctionTok{table}\NormalTok{(treino\_imp[[target\_var]]), }\AttributeTok{decreasing =} \ConstantTok{TRUE}\NormalTok{))[}\DecValTok{1}\NormalTok{]}
\NormalTok{treino\_ENN }\OtherTok{\textless{}{-}} \FunctionTok{ENN\_manual}\NormalTok{(treino\_imp, }\AttributeTok{target =}\NormalTok{ target\_var, }\AttributeTok{k =} \DecValTok{3}\NormalTok{, }\AttributeTok{majority\_class =}\NormalTok{ maj\_class)}

\NormalTok{prop\_ENN }\OtherTok{\textless{}{-}} \FunctionTok{prop.table}\NormalTok{(}\FunctionTok{table}\NormalTok{(treino\_ENN[[target\_var]]))}
\FunctionTok{kable}\NormalTok{(}\FunctionTok{as.data.frame}\NormalTok{(prop\_ENN), }\AttributeTok{col.names =} \FunctionTok{c}\NormalTok{(}\StringTok{"Clase"}\NormalTok{, }\StringTok{"Proporción"}\NormalTok{),}
\AttributeTok{caption =} \StringTok{"Proporciones en entrenamiento después de ENN"}\NormalTok{)}
\end{Highlighting}
\end{Shaded}

\begin{longtable}[]{@{}lr@{}}
\caption{Proporciones en entrenamiento después de ENN}\tabularnewline
\toprule\noalign{}
Clase & Proporción \\
\midrule\noalign{}
\endfirsthead
\toprule\noalign{}
Clase & Proporción \\
\midrule\noalign{}
\endhead
\bottomrule\noalign{}
\endlastfoot
0 & 0.9322034 \\
1 & 0.0677966 \\
\end{longtable}

\begin{Shaded}
\begin{Highlighting}[]
\NormalTok{cut\_ENN }\OtherTok{\textless{}{-}} \FunctionTok{as.numeric}\NormalTok{(}\FunctionTok{min}\NormalTok{(}\FunctionTok{prop.table}\NormalTok{(}\FunctionTok{table}\NormalTok{(treino\_imp[[target\_var]]))))}
\NormalTok{res\_d   }\OtherTok{\textless{}{-}} \FunctionTok{eval\_logit}\NormalTok{(treino\_ENN, teste\_imp, }\AttributeTok{cutoff =}\NormalTok{ cut\_ENN)}
\NormalTok{row\_d   }\OtherTok{\textless{}{-}} \FunctionTok{cbind}\NormalTok{(}\AttributeTok{Modelo =} \FunctionTok{sprintf}\NormalTok{(}\StringTok{"ENN (cut=\%.3f)"}\NormalTok{, cut\_ENN), }\AttributeTok{Ntrain =}\NormalTok{ res\_d}\SpecialCharTok{$}\NormalTok{n\_train, }\AttributeTok{Cutoff =}\NormalTok{ res\_d}\SpecialCharTok{$}\NormalTok{cutoff, res\_d}\SpecialCharTok{$}\NormalTok{metrics)}
\FunctionTok{kable}\NormalTok{(row\_d, }\AttributeTok{caption =} \StringTok{"Resultados (d): ENN + logística (cut = prop. minoritaria)"}\NormalTok{)}
\end{Highlighting}
\end{Shaded}

\begin{longtable}[]{@{}
  >{\raggedright\arraybackslash}p{(\linewidth - 20\tabcolsep) * \real{0.1368}}
  >{\raggedleft\arraybackslash}p{(\linewidth - 20\tabcolsep) * \real{0.0598}}
  >{\raggedleft\arraybackslash}p{(\linewidth - 20\tabcolsep) * \real{0.0855}}
  >{\raggedleft\arraybackslash}p{(\linewidth - 20\tabcolsep) * \real{0.0855}}
  >{\raggedleft\arraybackslash}p{(\linewidth - 20\tabcolsep) * \real{0.1026}}
  >{\raggedleft\arraybackslash}p{(\linewidth - 20\tabcolsep) * \real{0.1026}}
  >{\raggedleft\arraybackslash}p{(\linewidth - 20\tabcolsep) * \real{0.0855}}
  >{\raggedleft\arraybackslash}p{(\linewidth - 20\tabcolsep) * \real{0.0855}}
  >{\raggedleft\arraybackslash}p{(\linewidth - 20\tabcolsep) * \real{0.0855}}
  >{\raggedleft\arraybackslash}p{(\linewidth - 20\tabcolsep) * \real{0.0855}}
  >{\raggedleft\arraybackslash}p{(\linewidth - 20\tabcolsep) * \real{0.0855}}@{}}
\caption{Resultados (d): ENN + logística (cut = prop.
minoritaria)}\tabularnewline
\toprule\noalign{}
\begin{minipage}[b]{\linewidth}\raggedright
Modelo
\end{minipage} & \begin{minipage}[b]{\linewidth}\raggedleft
Ntrain
\end{minipage} & \begin{minipage}[b]{\linewidth}\raggedleft
Cutoff
\end{minipage} & \begin{minipage}[b]{\linewidth}\raggedleft
Accuracy
\end{minipage} & \begin{minipage}[b]{\linewidth}\raggedleft
Sensitivity
\end{minipage} & \begin{minipage}[b]{\linewidth}\raggedleft
Specificity
\end{minipage} & \begin{minipage}[b]{\linewidth}\raggedleft
PPV
\end{minipage} & \begin{minipage}[b]{\linewidth}\raggedleft
NPV
\end{minipage} & \begin{minipage}[b]{\linewidth}\raggedleft
Gmean
\end{minipage} & \begin{minipage}[b]{\linewidth}\raggedleft
F1
\end{minipage} & \begin{minipage}[b]{\linewidth}\raggedleft
MCC
\end{minipage} \\
\midrule\noalign{}
\endfirsthead
\toprule\noalign{}
\begin{minipage}[b]{\linewidth}\raggedright
Modelo
\end{minipage} & \begin{minipage}[b]{\linewidth}\raggedleft
Ntrain
\end{minipage} & \begin{minipage}[b]{\linewidth}\raggedleft
Cutoff
\end{minipage} & \begin{minipage}[b]{\linewidth}\raggedleft
Accuracy
\end{minipage} & \begin{minipage}[b]{\linewidth}\raggedleft
Sensitivity
\end{minipage} & \begin{minipage}[b]{\linewidth}\raggedleft
Specificity
\end{minipage} & \begin{minipage}[b]{\linewidth}\raggedleft
PPV
\end{minipage} & \begin{minipage}[b]{\linewidth}\raggedleft
NPV
\end{minipage} & \begin{minipage}[b]{\linewidth}\raggedleft
Gmean
\end{minipage} & \begin{minipage}[b]{\linewidth}\raggedleft
F1
\end{minipage} & \begin{minipage}[b]{\linewidth}\raggedleft
MCC
\end{minipage} \\
\midrule\noalign{}
\endhead
\bottomrule\noalign{}
\endlastfoot
ENN (cut=0.067) & 590 & 0.0666667 & 0.9496124 & 0.9333333 & 0.9506173 &
0.5384615 & 0.9956897 & 0.9419357 & 0.6829268 & 0.6871414 \\
\end{longtable}

ENN elimina mayoritarios ``ruidosos''. Con corte igual a la proporción
minoritaria se empuja el modelo a recuperar más positivos.

\section{(e) SMOTE + ENN + Regresión logística (corte
0.5)}\label{e-smote-enn-regresiuxf3n-loguxedstica-corte-0.5}

\begin{Shaded}
\begin{Highlighting}[]
\NormalTok{treino\_ENN2 }\OtherTok{\textless{}{-}} \FunctionTok{ENN\_manual}\NormalTok{(treino\_imp, }\AttributeTok{target =}\NormalTok{ target\_var, }\AttributeTok{k =} \DecValTok{3}\NormalTok{, }\AttributeTok{majority\_class =}\NormalTok{ maj\_class)}
\NormalTok{rec\_smote2  }\OtherTok{\textless{}{-}} \FunctionTok{recipe}\NormalTok{(}\FunctionTok{as.formula}\NormalTok{(}\FunctionTok{paste}\NormalTok{(target\_var, }\StringTok{"\textasciitilde{} ."}\NormalTok{)), }\AttributeTok{data =}\NormalTok{ treino\_ENN2) }\SpecialCharTok{\%\textgreater{}\%}
\FunctionTok{step\_smote}\NormalTok{(}\FunctionTok{all\_outcomes}\NormalTok{()) }\SpecialCharTok{\%\textgreater{}\%}
\FunctionTok{step\_zv}\NormalTok{(}\FunctionTok{all\_predictors}\NormalTok{())}
\NormalTok{smote2\_prep      }\OtherTok{\textless{}{-}} \FunctionTok{prep}\NormalTok{(rec\_smote2)}
\NormalTok{treino\_ENN\_SMOTE }\OtherTok{\textless{}{-}} \FunctionTok{bake}\NormalTok{(smote2\_prep, }\AttributeTok{new\_data =} \ConstantTok{NULL}\NormalTok{)}

\NormalTok{res\_e }\OtherTok{\textless{}{-}} \FunctionTok{eval\_logit}\NormalTok{(treino\_ENN\_SMOTE, teste\_imp, }\AttributeTok{cutoff =} \FloatTok{0.5}\NormalTok{)}
\NormalTok{row\_e }\OtherTok{\textless{}{-}} \FunctionTok{cbind}\NormalTok{(}\AttributeTok{Modelo =} \StringTok{"SMOTE+ENN (cut=0.5)"}\NormalTok{, }\AttributeTok{Ntrain =}\NormalTok{ res\_e}\SpecialCharTok{$}\NormalTok{n\_train, }\AttributeTok{Cutoff =}\NormalTok{ res\_e}\SpecialCharTok{$}\NormalTok{cutoff, res\_e}\SpecialCharTok{$}\NormalTok{metrics)}
\FunctionTok{kable}\NormalTok{(row\_e, }\AttributeTok{caption =} \StringTok{"Resultados (e): SMOTE+ENN + logística (cut=0.5)"}\NormalTok{)}
\end{Highlighting}
\end{Shaded}

\begin{longtable}[]{@{}
  >{\raggedright\arraybackslash}p{(\linewidth - 20\tabcolsep) * \real{0.1709}}
  >{\raggedleft\arraybackslash}p{(\linewidth - 20\tabcolsep) * \real{0.0598}}
  >{\raggedleft\arraybackslash}p{(\linewidth - 20\tabcolsep) * \real{0.0598}}
  >{\raggedleft\arraybackslash}p{(\linewidth - 20\tabcolsep) * \real{0.0855}}
  >{\raggedleft\arraybackslash}p{(\linewidth - 20\tabcolsep) * \real{0.1026}}
  >{\raggedleft\arraybackslash}p{(\linewidth - 20\tabcolsep) * \real{0.1026}}
  >{\raggedleft\arraybackslash}p{(\linewidth - 20\tabcolsep) * \real{0.0855}}
  >{\raggedleft\arraybackslash}p{(\linewidth - 20\tabcolsep) * \real{0.0769}}
  >{\raggedleft\arraybackslash}p{(\linewidth - 20\tabcolsep) * \real{0.0855}}
  >{\raggedleft\arraybackslash}p{(\linewidth - 20\tabcolsep) * \real{0.0855}}
  >{\raggedleft\arraybackslash}p{(\linewidth - 20\tabcolsep) * \real{0.0855}}@{}}
\caption{Resultados (e): SMOTE+ENN + logística (cut=0.5)}\tabularnewline
\toprule\noalign{}
\begin{minipage}[b]{\linewidth}\raggedright
Modelo
\end{minipage} & \begin{minipage}[b]{\linewidth}\raggedleft
Ntrain
\end{minipage} & \begin{minipage}[b]{\linewidth}\raggedleft
Cutoff
\end{minipage} & \begin{minipage}[b]{\linewidth}\raggedleft
Accuracy
\end{minipage} & \begin{minipage}[b]{\linewidth}\raggedleft
Sensitivity
\end{minipage} & \begin{minipage}[b]{\linewidth}\raggedleft
Specificity
\end{minipage} & \begin{minipage}[b]{\linewidth}\raggedleft
PPV
\end{minipage} & \begin{minipage}[b]{\linewidth}\raggedleft
NPV
\end{minipage} & \begin{minipage}[b]{\linewidth}\raggedleft
Gmean
\end{minipage} & \begin{minipage}[b]{\linewidth}\raggedleft
F1
\end{minipage} & \begin{minipage}[b]{\linewidth}\raggedleft
MCC
\end{minipage} \\
\midrule\noalign{}
\endfirsthead
\toprule\noalign{}
\begin{minipage}[b]{\linewidth}\raggedright
Modelo
\end{minipage} & \begin{minipage}[b]{\linewidth}\raggedleft
Ntrain
\end{minipage} & \begin{minipage}[b]{\linewidth}\raggedleft
Cutoff
\end{minipage} & \begin{minipage}[b]{\linewidth}\raggedleft
Accuracy
\end{minipage} & \begin{minipage}[b]{\linewidth}\raggedleft
Sensitivity
\end{minipage} & \begin{minipage}[b]{\linewidth}\raggedleft
Specificity
\end{minipage} & \begin{minipage}[b]{\linewidth}\raggedleft
PPV
\end{minipage} & \begin{minipage}[b]{\linewidth}\raggedleft
NPV
\end{minipage} & \begin{minipage}[b]{\linewidth}\raggedleft
Gmean
\end{minipage} & \begin{minipage}[b]{\linewidth}\raggedleft
F1
\end{minipage} & \begin{minipage}[b]{\linewidth}\raggedleft
MCC
\end{minipage} \\
\midrule\noalign{}
\endhead
\bottomrule\noalign{}
\endlastfoot
SMOTE+ENN (cut=0.5) & 1100 & 0.5 & 0.9457364 & 0.9333333 & 0.9465021 &
0.5185185 & 0.995671 & 0.9398946 & 0.6666667 & 0.6726085 \\
\end{longtable}

La combinación no necesariamente supera a SMOTE solo; si ENN elimina
casos informativos antes del balanceo, la señal puede debilitarse.

\section{(f) Base desbalanceada + Regresión logística (dos
cortes)}\label{f-base-desbalanceada-regresiuxf3n-loguxedstica-dos-cortes}

\begin{Shaded}
\begin{Highlighting}[]
\CommentTok{\# Sin SMOTE/ENN: solo imputación/normalización}

\NormalTok{res\_f1 }\OtherTok{\textless{}{-}} \FunctionTok{eval\_logit}\NormalTok{(treino\_imp, teste\_imp, }\AttributeTok{cutoff =} \FloatTok{0.5}\NormalTok{)}
\NormalTok{row\_f1 }\OtherTok{\textless{}{-}} \FunctionTok{cbind}\NormalTok{(}\AttributeTok{Modelo =} \StringTok{"Desbalanceada (cut=0.5)"}\NormalTok{, }\AttributeTok{Ntrain =}\NormalTok{ res\_f1}\SpecialCharTok{$}\NormalTok{n\_train, }\AttributeTok{Cutoff =}\NormalTok{ res\_f1}\SpecialCharTok{$}\NormalTok{cutoff, res\_f1}\SpecialCharTok{$}\NormalTok{metrics)}

\NormalTok{res\_f2 }\OtherTok{\textless{}{-}} \FunctionTok{eval\_logit}\NormalTok{(treino\_imp, teste\_imp, }\AttributeTok{cutoff =}\NormalTok{ minority\_prop)}
\NormalTok{row\_f2 }\OtherTok{\textless{}{-}} \FunctionTok{cbind}\NormalTok{(}\AttributeTok{Modelo =} \FunctionTok{sprintf}\NormalTok{(}\StringTok{"Desbalanceada (cut=\%.3f)"}\NormalTok{, minority\_prop), }\AttributeTok{Ntrain =}\NormalTok{ res\_f2}\SpecialCharTok{$}\NormalTok{n\_train, }\AttributeTok{Cutoff =}\NormalTok{ res\_f2}\SpecialCharTok{$}\NormalTok{cutoff, res\_f2}\SpecialCharTok{$}\NormalTok{metrics)}

\FunctionTok{kable}\NormalTok{(row\_f1, }\AttributeTok{caption =} \StringTok{"Resultados (f1): Desbalanceada + logística (cut=0.5)"}\NormalTok{)}
\end{Highlighting}
\end{Shaded}

\begin{longtable}[]{@{}
  >{\raggedright\arraybackslash}p{(\linewidth - 20\tabcolsep) * \real{0.1967}}
  >{\raggedleft\arraybackslash}p{(\linewidth - 20\tabcolsep) * \real{0.0574}}
  >{\raggedleft\arraybackslash}p{(\linewidth - 20\tabcolsep) * \real{0.0574}}
  >{\raggedleft\arraybackslash}p{(\linewidth - 20\tabcolsep) * \real{0.0820}}
  >{\raggedleft\arraybackslash}p{(\linewidth - 20\tabcolsep) * \real{0.0984}}
  >{\raggedleft\arraybackslash}p{(\linewidth - 20\tabcolsep) * \real{0.0984}}
  >{\raggedleft\arraybackslash}p{(\linewidth - 20\tabcolsep) * \real{0.0820}}
  >{\raggedleft\arraybackslash}p{(\linewidth - 20\tabcolsep) * \real{0.0820}}
  >{\raggedleft\arraybackslash}p{(\linewidth - 20\tabcolsep) * \real{0.0820}}
  >{\raggedleft\arraybackslash}p{(\linewidth - 20\tabcolsep) * \real{0.0820}}
  >{\raggedleft\arraybackslash}p{(\linewidth - 20\tabcolsep) * \real{0.0820}}@{}}
\caption{Resultados (f1): Desbalanceada + logística
(cut=0.5)}\tabularnewline
\toprule\noalign{}
\begin{minipage}[b]{\linewidth}\raggedright
Modelo
\end{minipage} & \begin{minipage}[b]{\linewidth}\raggedleft
Ntrain
\end{minipage} & \begin{minipage}[b]{\linewidth}\raggedleft
Cutoff
\end{minipage} & \begin{minipage}[b]{\linewidth}\raggedleft
Accuracy
\end{minipage} & \begin{minipage}[b]{\linewidth}\raggedleft
Sensitivity
\end{minipage} & \begin{minipage}[b]{\linewidth}\raggedleft
Specificity
\end{minipage} & \begin{minipage}[b]{\linewidth}\raggedleft
PPV
\end{minipage} & \begin{minipage}[b]{\linewidth}\raggedleft
NPV
\end{minipage} & \begin{minipage}[b]{\linewidth}\raggedleft
Gmean
\end{minipage} & \begin{minipage}[b]{\linewidth}\raggedleft
F1
\end{minipage} & \begin{minipage}[b]{\linewidth}\raggedleft
MCC
\end{minipage} \\
\midrule\noalign{}
\endfirsthead
\toprule\noalign{}
\begin{minipage}[b]{\linewidth}\raggedright
Modelo
\end{minipage} & \begin{minipage}[b]{\linewidth}\raggedleft
Ntrain
\end{minipage} & \begin{minipage}[b]{\linewidth}\raggedleft
Cutoff
\end{minipage} & \begin{minipage}[b]{\linewidth}\raggedleft
Accuracy
\end{minipage} & \begin{minipage}[b]{\linewidth}\raggedleft
Sensitivity
\end{minipage} & \begin{minipage}[b]{\linewidth}\raggedleft
Specificity
\end{minipage} & \begin{minipage}[b]{\linewidth}\raggedleft
PPV
\end{minipage} & \begin{minipage}[b]{\linewidth}\raggedleft
NPV
\end{minipage} & \begin{minipage}[b]{\linewidth}\raggedleft
Gmean
\end{minipage} & \begin{minipage}[b]{\linewidth}\raggedleft
F1
\end{minipage} & \begin{minipage}[b]{\linewidth}\raggedleft
MCC
\end{minipage} \\
\midrule\noalign{}
\endhead
\bottomrule\noalign{}
\endlastfoot
Desbalanceada (cut=0.5) & 600 & 0.5 & 0.9457364 & 0.6666667 & 0.962963 &
0.5263158 & 0.9790795 & 0.8012336 & 0.5882353 & 0.5641027 \\
\end{longtable}

\begin{Shaded}
\begin{Highlighting}[]
\FunctionTok{kable}\NormalTok{(row\_f2, }\AttributeTok{caption =} \StringTok{"Resultados (f2): Desbalanceada + logística (cut=prop. minoritaria)"}\NormalTok{)}
\end{Highlighting}
\end{Shaded}

\begin{longtable}[]{@{}
  >{\raggedright\arraybackslash}p{(\linewidth - 20\tabcolsep) * \real{0.2047}}
  >{\raggedleft\arraybackslash}p{(\linewidth - 20\tabcolsep) * \real{0.0551}}
  >{\raggedleft\arraybackslash}p{(\linewidth - 20\tabcolsep) * \real{0.0787}}
  >{\raggedleft\arraybackslash}p{(\linewidth - 20\tabcolsep) * \real{0.0787}}
  >{\raggedleft\arraybackslash}p{(\linewidth - 20\tabcolsep) * \real{0.0945}}
  >{\raggedleft\arraybackslash}p{(\linewidth - 20\tabcolsep) * \real{0.0945}}
  >{\raggedleft\arraybackslash}p{(\linewidth - 20\tabcolsep) * \real{0.0787}}
  >{\raggedleft\arraybackslash}p{(\linewidth - 20\tabcolsep) * \real{0.0787}}
  >{\raggedleft\arraybackslash}p{(\linewidth - 20\tabcolsep) * \real{0.0787}}
  >{\raggedleft\arraybackslash}p{(\linewidth - 20\tabcolsep) * \real{0.0787}}
  >{\raggedleft\arraybackslash}p{(\linewidth - 20\tabcolsep) * \real{0.0787}}@{}}
\caption{Resultados (f2): Desbalanceada + logística (cut=prop.
minoritaria)}\tabularnewline
\toprule\noalign{}
\begin{minipage}[b]{\linewidth}\raggedright
Modelo
\end{minipage} & \begin{minipage}[b]{\linewidth}\raggedleft
Ntrain
\end{minipage} & \begin{minipage}[b]{\linewidth}\raggedleft
Cutoff
\end{minipage} & \begin{minipage}[b]{\linewidth}\raggedleft
Accuracy
\end{minipage} & \begin{minipage}[b]{\linewidth}\raggedleft
Sensitivity
\end{minipage} & \begin{minipage}[b]{\linewidth}\raggedleft
Specificity
\end{minipage} & \begin{minipage}[b]{\linewidth}\raggedleft
PPV
\end{minipage} & \begin{minipage}[b]{\linewidth}\raggedleft
NPV
\end{minipage} & \begin{minipage}[b]{\linewidth}\raggedleft
Gmean
\end{minipage} & \begin{minipage}[b]{\linewidth}\raggedleft
F1
\end{minipage} & \begin{minipage}[b]{\linewidth}\raggedleft
MCC
\end{minipage} \\
\midrule\noalign{}
\endfirsthead
\toprule\noalign{}
\begin{minipage}[b]{\linewidth}\raggedright
Modelo
\end{minipage} & \begin{minipage}[b]{\linewidth}\raggedleft
Ntrain
\end{minipage} & \begin{minipage}[b]{\linewidth}\raggedleft
Cutoff
\end{minipage} & \begin{minipage}[b]{\linewidth}\raggedleft
Accuracy
\end{minipage} & \begin{minipage}[b]{\linewidth}\raggedleft
Sensitivity
\end{minipage} & \begin{minipage}[b]{\linewidth}\raggedleft
Specificity
\end{minipage} & \begin{minipage}[b]{\linewidth}\raggedleft
PPV
\end{minipage} & \begin{minipage}[b]{\linewidth}\raggedleft
NPV
\end{minipage} & \begin{minipage}[b]{\linewidth}\raggedleft
Gmean
\end{minipage} & \begin{minipage}[b]{\linewidth}\raggedleft
F1
\end{minipage} & \begin{minipage}[b]{\linewidth}\raggedleft
MCC
\end{minipage} \\
\midrule\noalign{}
\endhead
\bottomrule\noalign{}
\endlastfoot
Desbalanceada (cut=0.064) & 600 & 0.0641026 & 0.9379845 & 0.9333333 &
0.9382716 & 0.4827586 & 0.9956332 & 0.9357992 & 0.6363636 & 0.6457311 \\
\end{longtable}

Tabla comparativa final

\begin{Shaded}
\begin{Highlighting}[]
\NormalTok{tabla\_final }\OtherTok{\textless{}{-}}\NormalTok{ dplyr}\SpecialCharTok{::}\FunctionTok{bind\_rows}\NormalTok{(row\_c, row\_d, row\_e, row\_f1, row\_f2) }\SpecialCharTok{\%\textgreater{}\%}
\FunctionTok{as\_tibble}\NormalTok{()}

\FunctionTok{kable}\NormalTok{(tabla\_final, }\AttributeTok{digits =} \DecValTok{3}\NormalTok{, }\AttributeTok{caption =} \StringTok{"Resumen de desempeño por estrategia"}\NormalTok{)}
\end{Highlighting}
\end{Shaded}

\begin{longtable}[]{@{}
  >{\raggedright\arraybackslash}p{(\linewidth - 20\tabcolsep) * \real{0.2524}}
  >{\raggedleft\arraybackslash}p{(\linewidth - 20\tabcolsep) * \real{0.0680}}
  >{\raggedleft\arraybackslash}p{(\linewidth - 20\tabcolsep) * \real{0.0680}}
  >{\raggedleft\arraybackslash}p{(\linewidth - 20\tabcolsep) * \real{0.0874}}
  >{\raggedleft\arraybackslash}p{(\linewidth - 20\tabcolsep) * \real{0.1165}}
  >{\raggedleft\arraybackslash}p{(\linewidth - 20\tabcolsep) * \real{0.1165}}
  >{\raggedleft\arraybackslash}p{(\linewidth - 20\tabcolsep) * \real{0.0583}}
  >{\raggedleft\arraybackslash}p{(\linewidth - 20\tabcolsep) * \real{0.0583}}
  >{\raggedleft\arraybackslash}p{(\linewidth - 20\tabcolsep) * \real{0.0583}}
  >{\raggedleft\arraybackslash}p{(\linewidth - 20\tabcolsep) * \real{0.0583}}
  >{\raggedleft\arraybackslash}p{(\linewidth - 20\tabcolsep) * \real{0.0583}}@{}}
\caption{Resumen de desempeño por estrategia}\tabularnewline
\toprule\noalign{}
\begin{minipage}[b]{\linewidth}\raggedright
Modelo
\end{minipage} & \begin{minipage}[b]{\linewidth}\raggedleft
Ntrain
\end{minipage} & \begin{minipage}[b]{\linewidth}\raggedleft
Cutoff
\end{minipage} & \begin{minipage}[b]{\linewidth}\raggedleft
Accuracy
\end{minipage} & \begin{minipage}[b]{\linewidth}\raggedleft
Sensitivity
\end{minipage} & \begin{minipage}[b]{\linewidth}\raggedleft
Specificity
\end{minipage} & \begin{minipage}[b]{\linewidth}\raggedleft
PPV
\end{minipage} & \begin{minipage}[b]{\linewidth}\raggedleft
NPV
\end{minipage} & \begin{minipage}[b]{\linewidth}\raggedleft
Gmean
\end{minipage} & \begin{minipage}[b]{\linewidth}\raggedleft
F1
\end{minipage} & \begin{minipage}[b]{\linewidth}\raggedleft
MCC
\end{minipage} \\
\midrule\noalign{}
\endfirsthead
\toprule\noalign{}
\begin{minipage}[b]{\linewidth}\raggedright
Modelo
\end{minipage} & \begin{minipage}[b]{\linewidth}\raggedleft
Ntrain
\end{minipage} & \begin{minipage}[b]{\linewidth}\raggedleft
Cutoff
\end{minipage} & \begin{minipage}[b]{\linewidth}\raggedleft
Accuracy
\end{minipage} & \begin{minipage}[b]{\linewidth}\raggedleft
Sensitivity
\end{minipage} & \begin{minipage}[b]{\linewidth}\raggedleft
Specificity
\end{minipage} & \begin{minipage}[b]{\linewidth}\raggedleft
PPV
\end{minipage} & \begin{minipage}[b]{\linewidth}\raggedleft
NPV
\end{minipage} & \begin{minipage}[b]{\linewidth}\raggedleft
Gmean
\end{minipage} & \begin{minipage}[b]{\linewidth}\raggedleft
F1
\end{minipage} & \begin{minipage}[b]{\linewidth}\raggedleft
MCC
\end{minipage} \\
\midrule\noalign{}
\endhead
\bottomrule\noalign{}
\endlastfoot
SMOTE (cut=0.5) & 1120 & 0.500 & 0.953 & 0.933 & 0.955 & 0.560 & 0.996 &
0.944 & 0.700 & 0.702 \\
ENN (cut=0.067) & 590 & 0.067 & 0.950 & 0.933 & 0.951 & 0.538 & 0.996 &
0.942 & 0.683 & 0.687 \\
SMOTE+ENN (cut=0.5) & 1100 & 0.500 & 0.946 & 0.933 & 0.947 & 0.519 &
0.996 & 0.940 & 0.667 & 0.673 \\
Desbalanceada (cut=0.5) & 600 & 0.500 & 0.946 & 0.667 & 0.963 & 0.526 &
0.979 & 0.801 & 0.588 & 0.564 \\
Desbalanceada (cut=0.064) & 600 & 0.064 & 0.938 & 0.933 & 0.938 & 0.483
& 0.996 & 0.936 & 0.636 & 0.646 \\
\end{longtable}

\section{(g) Discusión y
conclusión}\label{g-discusiuxf3n-y-conclusiuxf3n}

Criterios robustos en desbalance. En datos desbalanceados, F1, G-mean y
MCC son métricas más informativas que la exactitud. F1 resume
precisión--recobrado, G-mean evalúa el balance
sensibilidad--especificidad y MCC mide la correlación entre predicción y
verdad (--1 a 1), siendo muy exigente.

\begin{Shaded}
\begin{Highlighting}[]
\CommentTok{\# Elegimos el "mejor" por F1 y, como desempate, por MCC y G{-}mean}

\NormalTok{pick }\OtherTok{\textless{}{-}}\NormalTok{ tabla\_final }\SpecialCharTok{\%\textgreater{}\%}
\FunctionTok{mutate}\NormalTok{(}\FunctionTok{across}\NormalTok{(}\FunctionTok{c}\NormalTok{(F1, MCC, Gmean), as.numeric)) }\SpecialCharTok{\%\textgreater{}\%}
\FunctionTok{arrange}\NormalTok{(}\FunctionTok{desc}\NormalTok{(F1), }\FunctionTok{desc}\NormalTok{(MCC), }\FunctionTok{desc}\NormalTok{(Gmean)) }\SpecialCharTok{\%\textgreater{}\%}
\FunctionTok{slice}\NormalTok{(}\DecValTok{1}\NormalTok{)}
\NormalTok{best\_model   }\OtherTok{\textless{}{-}}\NormalTok{ pick}\SpecialCharTok{$}\NormalTok{Modelo}
\NormalTok{best\_f1      }\OtherTok{\textless{}{-}}\NormalTok{ pick}\SpecialCharTok{$}\NormalTok{F1}
\NormalTok{best\_mcc     }\OtherTok{\textless{}{-}}\NormalTok{ pick}\SpecialCharTok{$}\NormalTok{MCC}
\NormalTok{best\_gmean   }\OtherTok{\textless{}{-}}\NormalTok{ pick}\SpecialCharTok{$}\NormalTok{Gmean}
\end{Highlighting}
\end{Shaded}

\section{Conclusión principal.}\label{conclusiuxf3n-principal.}

La estrategia con mejor compromiso global fue r best\_model, con F1 = r
sprintf(`\%.3f', best\_f1), MCC = r sprintf(`\%.3f', best\_mcc) y G-mean
= r sprintf(`\%.3f', best\_gmean). Esto indica que logra detectar más
positivos sin deteriorar en exceso la precisión ni la especificidad, y
mantiene una correlación alta entre las predicciones y la realidad.

\subsection{Influencia del punto de
corte.}\label{influencia-del-punto-de-corte.}

Con corte = 0.5 en base desbalanceada, el clasificador favorece la clase
mayoritaria (alta especificidad, baja sensibilidad).

Con corte = proporción minoritaria, aumenta la sensibilidad (recupera
más casos positivos) a costa de cierta pérdida en precisión.

Tras SMOTE, un corte = 0.5 vuelve razonable porque el entrenamiento está
balanceado; de allí el buen F1/G-mean/MCC observados.

\subsection{Qué usaría en
producción.}\label{quuxe9-usaruxeda-en-producciuxf3n.}

Adoptaría SMOTE + logística (con pipeline de imputación kNN →
normalización, y chequeo de predictores constantes), dejando el punto de
corte calibrado en validación según la métrica prioritaria clínica
(p.~ej., maximizar F1 o un costo asimétrico de falso negativo).

ENN puede ser útil cuando hay mucho ruido en la mayoría, pero en estos
datos no superó a SMOTE solo.

El preprocesamiento importa: con datos clínicos desbalanceados y con NA,
la imputación multivariada, el balanceo y la calibración del umbral
hacen la diferencia entre un modelo ``con buena exactitud'' y uno
realmente útil para detectar casos positivos.

\end{document}
